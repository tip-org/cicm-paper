\documentclass{llncs}
\usepackage{tikz}
\usetikzlibrary{shapes,arrows,calc}


%\usepackage{url}
\usepackage{multirow}
\usepackage{listings}
\usepackage{amsmath}  % for equation*
\usepackage{array}    % for tabular
\usepackage{verbatim} % for comment
\usepackage{wrapfig}
\usepackage[final]{microtype}
\usepackage[pdfborder={0 0 0}]{hyperref}
\usepackage{tabularx}

\newcommand\forAll[1]{\forall \, #1 \, . \,}
\newcommand\forAllII[2]{\forall \, #1 \, #2 \, . \,}

\newcommand\propno[1]{(\emph{#1})}
\newcommand\hs[1]{\texttt{#1}}

% \raggedbottom

\lstnewenvironment{code}[1][]
  {\noindent
   \vspace{-0.5\baselineskip}
   \lstset{basicstyle=\ttfamily,
           frame=single,
           language=Haskell,
           keywordstyle=\color{black},
           #1}
   \fontsize{8pt}{8pt}\selectfont}
  {}

\newcommand\NOTE[1]{} % \mbox{}\marginpar{\fontsize{8pt}{8pt}\selectfont\raggedright\hspace{0pt}\emph{#1}}}

\def\maindocument{} % To tell tikz images that they are not stand alone

\begin{document}

\title{TIP: Tons of Inductive Problems}

\author{Moa Johansson \and Dan Ros\'en \and Nicholas Smallbone \and Koen Claessen}
\institute{Department of Computer Science and Engineering, Chalmers University of Technology
	\email{\{jomoa,danr,nicsma,koen\}@chalmers.se}
	}

\authorrunning{Johansson, Ros\'en, Smallbone, Claessen}
\titlerunning{Integrating Theory Exploration in a Proof Assistant}

\maketitle

\begin{abstract}

New testsuite (format)!
\end{abstract}

\section{Introduction}

goal: a format with explicit function definitions (hipspec, zeno, isaplanner, acl2, dafny),
      but also usable for provers that are not oriented towards functional programs (cvc4, new spass)
      (but still use explicit datatype definitions (?)).

\section{The TIP language/format}

based on SMT-LIB
  differences:
    recursive define-fun (note: http://stackoverflow.com/questions/7740556/equivalent-of-define-fun-in-z3-api#7745041)
    polymorphism
    match-expression
    + prove top level statement
    + define-datatype instead of define-datatypes
         (due to mutually recursive datatypes might differ in kind because of polymorphism)
  advantages:
    can import well defined theories from SMTLIB, such as linear arithmetic
    easy to parse

bnf + example

\section{The benchmarks}

https://github.com/tip-org/benchmarks

from the hipspec-cade paper. two testsuites: overlap?

+ our own examples

+ koen's new examples

all functions in the benchmarks terminate. the format does not presuppose this

\section{Tools}

done: ghc frontend (subset of Haskell). example

done: lambda lifting...

could be done before deadline: parser. tip -> smtlib.

before camera ready deadline (work in progress): monomorphiser, tip -> other formats

\section{Summary}

\bibliographystyle{plain}
\bibliography{bibfile}

\end{document}
