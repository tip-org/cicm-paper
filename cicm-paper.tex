\documentclass{llncs}
\usepackage{tikz}
\usetikzlibrary{shapes,arrows,calc}


%\usepackage{url}
\usepackage{multirow}
\usepackage{listings}
\usepackage{amsmath}  % for equation*
\usepackage{array}    % for tabular
\usepackage{verbatim} % for comment
\usepackage{wrapfig}
\usepackage[final]{microtype}
\usepackage[pdfborder={0 0 0}]{hyperref}
\usepackage{tabularx}

\newcommand\forAll[1]{\forall \, #1 \, . \,}
\newcommand\forAllII[2]{\forall \, #1 \, #2 \, . \,}

\newcommand\propno[1]{(\emph{#1})}
\newcommand\hs[1]{\texttt{#1}}

% \raggedbottom

\lstnewenvironment{code}[1][]
  {\noindent
   \vspace{-0.5\baselineskip}
   \lstset{basicstyle=\ttfamily,
           frame=single,
           language=Haskell,
           keywordstyle=\color{black},
           #1}
   \fontsize{8pt}{8pt}\selectfont}
  {}

\newcommand\NOTE[1]{} % \mbox{}\marginpar{\fontsize{8pt}{8pt}\selectfont\raggedright\hspace{0pt}\emph{#1}}}

\def\maindocument{} % To tell tikz images that they are not stand alone

\begin{document}

\title{TIP: Tons of Inductive Problems}

\author{Moa Johansson \and Dan Ros\'en \and Nicholas Smallbone \and Koen Claessen}
\institute{Department of Computer Science and Engineering, Chalmers University of Technology
	\email{\{jomoa,danr,nicsma,koen\}@chalmers.se}
	}

\authorrunning{Johansson, Ros\'en, Smallbone, Claessen}
\titlerunning{}

\maketitle

\begin{abstract}
There has recently been increased interest in inductive theorem
proving, both in special-purpose provers such as
IsaPlanner, Zeno and HipSpec, and in SMT-solvers such as
Dafny/Z3 and CVC4. To ease evaluation and comparison between systems,
good benchmarks are important. However, there has not been a shared
standard benchmark suite for inductive theorem provers. This paper
describes initial efforts to collect benchmarks for inductive theorem
provers which can be shared among developers and users. 
We expect this benchmark suite to continuously grow as
more problems are submitted by the community. New challenge problems
will promote new developments of provers which will greatly benefit
both developers and users of inductive theorem provers. 

\end{abstract}

\section{Introduction}

The number of inductive theorem provers has been increasing, both with specialised provers such as IsaPlanner, Zeno and HipSpec \cite{isaplanner,zeno,hipspecCADE}, SMT-solvers such as Dafny/Z3 and CVC4 \cite{dafny,cvc4}, recent work on first-order SPASS prover \cite{SPASSInduction} as well as some support in proof assistants \cite{acl2,hipster}. 

To ease evaluation, development and compare the relative strengths of the different systems it is important to have good, standard benchmarks. In this paper we describe initial efforts to collect such benchmarks for inductive theorem provers. The benchmarks are publicly available at \url{https://github.com/tip-org/benchmarks}. We invite the community to submit additional problems and challenges and expect this benchmark suite to continuously grow and encourage further development of inductive theorem provers. 

The benchmarks are currently expressed in a subset of the WhyML specification language \cite{whyML}. We choose this format as the Why3 system already support translation tools from WhyML to various other formats, such as SMT-LIB and versions of TPTP. In addition, it is human readable and has a range of desirable properties which we describe in \S \ref{sec:format}. 

\section{The Benchmark Format}
\label{sec:format}

\section{Sample Benchmarks}

The benchmarks are publicly available at \url{https://github.com/tip-org/benchmarks}. 

Benchmarks are written in a subset of the WhyML language, with the recent extension to support higher-order functions.

Which tools/translationgs are available for WhyML?

How to contribute benchmarks?

from the hipspec-cade paper. two testsuites: overlap?

+ our own examples

+ koen's new examples

all functions in the benchmarks terminate. the format does not presuppose this

%\section{Tools}
%
%done: ghc frontend (subset of Haskell). example
%
%done: lambda lifting...
%
%could be done before deadline: parser. tip -> smtlib.
%
%before camera ready deadline (work in progress): monomorphiser, tip -> other formats

\section{Conclusion and Further Work}
- Tools for conversion to other formats

- Solicit submissions of additional benchmarks and challenge problems.

\bibliographystyle{plain}
\bibliography{bibfile}

\end{document}
