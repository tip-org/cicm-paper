\documentclass{llncs}
\usepackage{tikz}
\usetikzlibrary{shapes,arrows,calc}


%\usepackage{url}
\usepackage{multirow}
\usepackage{listings}
\usepackage{amsmath}  % for equation*
\usepackage{array}    % for tabular
\usepackage{verbatim} % for comment
\usepackage{wrapfig}
\usepackage[final]{microtype}
\usepackage[pdfborder={0 0 0}]{hyperref}
\usepackage{tabularx}

\newcommand\forAll[1]{\forall \, #1 \, . \,}
\newcommand\forAllII[2]{\forall \, #1 \, #2 \, . \,}

\newcommand\propno[1]{(\emph{#1})}
\newcommand\hs[1]{\texttt{#1}}

% \raggedbottom

\lstnewenvironment{code}[1][]
  {\noindent
   \vspace{-0.5\baselineskip}
   \lstset{basicstyle=\ttfamily,
           frame=single,
           language=Haskell,
           keywordstyle=\color{black},
           #1}
   \fontsize{8pt}{8pt}\selectfont}
  {}

\newcommand\NOTE[1]{} % \mbox{}\marginpar{\fontsize{8pt}{8pt}\selectfont\raggedright\hspace{0pt}\emph{#1}}}

\def\maindocument{} % To tell tikz images that they are not stand alone

\begin{document}

\title{TIP: Tons of Inductive Problems}

\author{Koen Claessen \and Moa Johansson \and Dan Ros\'en \and Nicholas Smallbone}
\institute{Department of Computer Science and Engineering, Chalmers University of Technology
	\email{\{koen,jomoa,danr,nicsma\}@chalmers.se}
	}

\authorrunning{Claessen, Johansson, Ros\'en, Smallbone}
\titlerunning{}

\maketitle

\begin{abstract} This paper describes our initial efforts to 
collect benchmarks for inductive theorem provers which can be 
shared among developers and users. The recent increase of 
interest in automated inductive theorem proving has also 
increased the demands for evaluation and comparison between 
systems. Our benchmark suite aims to become the standard shared 
benchmark suite for inductive theorem provers. We expect the 
benchmark suite to continuously grow as more problems are 
submitted by the community. New challenge problems will promote 
new developments of provers which will greatly benefit both 
developers and users of inductive theorem provers. 
\end{abstract}

\section{Introduction}

We have recently seen increased interest in inductive theorem 
proving, both with specialised provers such as IsaPlanner, Zeno 
and HipSpec \cite{hipspecCADE,dixon2007isaplanner,zeno}, 
SMT-solvers such as CVC4 \cite{cvc4}, the auto-active prover 
Dafny \cite{dafny}, recent work on the first-order SPASS prover 
\cite{SPASSInduction}, as well as some support in proof 
assistants \cite{hipster,acl2}.

To ease evaluation and development, and compare the relative 
strengths of the different systems, it is important to have good 
standard benchmarks. The contribution of this paper is an 
accessible standard benchmark suite for inductive theorem 
provers which can be extended by users and developers. The 
benchmarks are publicly available at:
\begin{center}
\url{https://github.com/tip-org/benchmarks}
\end{center}
We have so far collected 340 problems in our benchmark suite,
which we have called ``TIP'', for Tons of Inductive Problems --
so named in the hope of attracting many more problems! We invite
the community to submit additional problems and challenges and
expect the collection to continuously grow and provide new
challenges for developers.

%% We do not yet discuss any special-purpose language features for expressing inductive problems in general. The benchmarks are therefore currently expressed in a subset of the WhyML specification language from the Why3 program verification system \cite{boogie11why3,Why3}. We have chosen this format as the Why3 system already provides translation tools from WhyML to various other common formats, such as SMT-LIB and versions of TPTP.
%% In addition, the WhyML language is also easy for humans to read. The Why3 system relies on external provers to discharge proof obligations and is already connected to a wide variety of automated and interactive provers, including  Alt-Ergo, CVC3, Z3, E, SPASS, Vampire, Coq and PVS. We hope that the new benchmark suite will encourage the connection of automated inductive theorem provers to Why3.

\section{The Benchmark Format}
\label{sec:format}
The benchmarks are expressed in a variant of SMT-LIB \cite{BST10},
extended with support for algebraic datatypes and higher-order
functions. The format is described in detail in \cite{tiplang}.
Starting from the basic SMT-LIB format, we import the following
features from existing systems:
\begin{itemize}
\item Algebraic datatypes, which are declared using the
  \verb|declare-datatypes| syntax as supported in Z3 and CVC4.
\item Recursive function definitions, which use the
  \verb|define-funs-rec| syntax implemented in CVC4 and proposed for
  SMT-LIB 2.5 \cite{BST10}.
\item Polymorphic functions, using the proposed \verb|par| syntax
  \cite{cvc4parPR}, which is implemented in a version of CVC4.
\end{itemize}

We then add more features of our own, which are specific to TIP:
\begin{itemize}
\item In the standard \verb|declare-datatypes| syntax, functions over
  algebraic datatypes are defined using projection functions like
  \verb|head| and \verb|tail|. We add a pattern-matching syntax,
  which is more convenient for many provers.
\item Many of our benchmark problems use higher-order functions,
  such as \verb|map|. We add syntax for lambda functions and
  higher-order functions.
\item Many inductive provers treat the goal specially, as opposed to
  SMT-LIB which expresses the goal as a negated axiom. We add a
  construct \verb|(assert-not p)| which declares \verb|p| as the goal;
  it is semantically equivalent to \verb|(assert (not p))|.
\end{itemize}

We do not necessarily expect every prover to support TIP 
natively. Instead, we have made a tool which can translate 
TIP problems to and from a variety of other formats. Currently 
our tool can translate TIP problems to a CVC4-compatible version 
of SMT-LIB or to WhyML, and can compile Haskell programs into 
TIP properties. It can also perform a number of transformations 
for tools which do not support the full TIP language, such as 
removing higher-order functions by defunctionalisation 
\cite{defunc}. Our aim is to support many different source and
target formats in this tool.

\subsection*{Example}
As an example of what the benchmarks look like, we show 
\texttt{prop12} from the IsaPlanner benchmark set (see 
Section~\ref{sec:isap} below), which states that the functions 
\texttt{drop} and \texttt{map} distribute over one another: 
\begin{center}
\texttt{drop n (map f xs) = map f (drop n xs)}.
\end{center}
We declare two simple algebraic datatypes representing natural 
numbers and polymorphic lists.

\begin{code}
(declare-datatypes (a)
  ((list (nil) (cons (head a) (tail (list a))))))
(declare-datatypes () ((Nat (Z) (S (p Nat)))))
\end{code}

Next, we declare two recursive functions: \texttt{drop}, which 
recursively drops a given number of elements from the front of 
the list and \texttt{map}, which is a higher-order function 
applying the given unary function \texttt{f} to each element of 
a list.

\begin{code}
(define-funs-rec
  ((par (a b) (map ((f (=> a b)) (xs (list a))) (list b))))
  ((match xs
     (case nil (as nil (list b)))
     (case (cons y ys) (cons (@ f y) (map f ys))))))
(define-funs-rec
  ((par (a) (drop ((n Nat) (xs (list a))) (list a))))
  ((match n
     (case Z xs)
     (case (S m)
       (match xs
         (case nil xs)
         (case (cons y ys) (drop m ys)))))))
\end{code}

The definition illustrates several features of TIP:
\begin{itemize}
  \item Polymorphism: \verb|par| introduces type variables.
  \item Higher-order functions: \verb|(=> a b)| is the type of
    functions from \verb|a| to \verb|b|, while \verb|@| applies
    a first-class function to its argument.
  \item Pattern-matching.
\end{itemize}

Finally, the benchmark problem itself is declared with the keyword \texttt{assert-not}:

\begin{code}
(assert-not
  (par (a b)
    (forall ((n Nat) (f (=> a b)) (xs (list a)))
      (= (drop n (map f xs)) (map f (drop n xs))))))
(check-sat)
\end{code}
Each benchmark file is stand-alone and only contains one property.

\section{Sample Benchmarks}
In this section we give a short overview of some the benchmark problems currently available in the repository. At the moment, there are three different main sources of problems. We expect more to be added.

\subsection{IsaPlanner's Rippling and Case-Analysis Benchmarks}
\label{sec:isap}
This set of 85 problems comes from the evaluation of IsaPlanner's rippling-heuristic for guiding rewriting in inductive proofs in the context of functions with case- and if-statements \cite{IsaPcase}. It has been used in the evaluation of many of the recent inductive theorem provers and includes theorems about lists, natural numbers and binary trees. The problems are relatively easy, most of the theorems can be proved by structural induction using only the function definitions and only 15 require auxiliary lemmas to be discovered.

\subsection{Productive Use of Failure Benchmarks}
This is another benchmark suite which has been used to evaluate several recent provers. It consists of 50 theorems about lists and natural numbers and originates from evaluation of techniques for discovering auxiliary lemmas in the CLAM prover \cite{productiveuse}. The original paper did not provide definitions for the functions used in the benchmarks, so the definitions provided here come from the evaluation of the HipSpec system \cite{hipspecCADE}. These proofs are generally a bit harder, and may require additional lemmas to be found and proved (by another induction) or generalisation of the conjecture in order to strengthen the inductive hypothesis.

\subsection{New TIP Benchmarks}
This set contains 205 new benchmarks including, amongst others,
properties of the Agda standard
library\footnote{\url{https://github.com/agda/agda-stdlib}}
implementation of integers on top of natural numbers, problems about natural numbers in binary representation, various sorting functions with correctness properties expressed in alternative ways, problems about regular expressions, binary search trees, grammars and skew heaps. The problems about sorting, regular expressions, grammars and heaps have to our knowledge not all been fully automated yet and are offered as challenges!

\section{Contribute to TIP}
We invite the theorem proving community to contribute additional inductive
benchmarks and challenge problems to TIP. Instructions for how to submit
problems can be found in the README file for the repository
(\url{https://github.com/tip-org/benchmarks}.)

We are developing a toolchain for translating to and from our format. The
development is in its own repository
(\url{https://github.com/tip-org/tools}).
The tool can currently read in problems in TIP format or Haskell, and
output TIP, SMT-LIB and WhyML, with some caveats:
\begin{itemize}
  \item The generated SMT-LIB uses \verb|declare-datatypes|,
    \verb|define-funs-rec| and polymorphism.
  \item The generated WhyML makes no special effort to pass Why3's
    termination checker.
\end{itemize}
We encourage the community to request and contribute additional input
and output formats to our toolchain.

\section{Conclusion and Further Work}

TIP is intended to be a standard benchmark suite for developers 
and users of inductive theorem provers. We hope that this 
initiative will ease comparison and evaluation of systems and 
spur further collaboration and development by attracting 
submissions of additional challenge problems from the community.

In addition to serving as a standard benchmark suite for 
inductive provers, the TIP benchmarks may also be useful for 
developers of theory exploration systems. Theory exploration is 
a technique for automatically discovering interesting 
conjectures about a given set of functions and datatypes, and is 
used in for example the HipSpec prover to discover lemmas. The 
TIP benchmarks can be compared to the output from the theory 
explorer in precision/recall analysis to assess the quality and 
interestingness of the conjectures generated. A good theory 
exploration system may also be used to generate new benchmarks 
for TIP.

It is important to have good tool support if TIP is to be used 
by the community. Our tool is currently in active development, 
in order to support more input and output formats, as well as 
various strategies for encoding features of TIP for provers that 
do not support the full language.

%The TIP benchmarks are currently expressed in a subset of WhyML. If we want to use the Why3 translation tools to other formats, there are some restrictions. Functions are, for instance, required to be provably terminating by the Why3 termination checker, which might not always be the case (e.g., it fails to prove termination for many functions that are not structurally recursive).
%In the future, we therefore expect to support a richer language, and will provide tools to help developers use the TIP format.

In the future, we may want to extend the language to support a 
richer variety of problems. For example, we may want to include 
problems about lazy functions, co-datatypes, and functions that 
possibly do not terminate.

%We may in the future perhaps also want to include problems featuring functions in a programming language with lazy evaluation.

Another aim is to also support non-theorems, for evaluation of 
tools for counter-example finding. This requires no extension to 
the format at all, but it requires a standardization on how to 
annotate problem with their expected answer, as well as a common 
solution format.

\bibliographystyle{plain}
\bibliography{bibfile}

\end{document}
